\documentclass[12pt,reqno]{amsart}


\usepackage{enumerate}
\usepackage{mathrsfs,hyperref}
\usepackage{amsfonts}
\usepackage{epsfig,amsfonts,amsmath}
\usepackage{graphicx}
\usepackage{mathabx}
\usepackage{amsthm}
% \usepackage[doublespacing]{setspace}
\usepackage{textcomp}
\usepackage{amsmath,amssymb}
\usepackage{fancyhdr}
% \usepackage{lastpage}
\usepackage{booktabs}
\usepackage{makecell}
\usepackage{tabularx}
\usepackage{mathrsfs}
\usepackage{ragged2e}
\usepackage{bm}
\usepackage{float}
\usepackage{color}
\usepackage{dynkin-diagrams}
\usepackage{indentfirst}
\usepackage{blkarray}
% \usepackage[square]{natbib}

\usepackage{geometry}

\geometry{a4paper,left=3cm,right=2.5cm,top=2.5cm,bottom=2cm}

%\usepackage{setspace}

%\renewcommand{\baselinestretch}{1.5}%行间距1.5倍




% \renewcommand{\contentsname}{\fontsize{18pt}{\baselineskip}\selectfont \textbf{Contents}}%Contents 18pt





\title[On intersecting maximal subgroups]{On intersecting maximal subgroups}




\date{\today}

\newcommand{\id}{\operatorname{id}}
\newcommand{\Aut}{\operatorname{Aut}}
\newcommand{\Inn}{\operatorname{Inn}}
\newcommand{\Out}{\operatorname{Out}}
\newcommand{\soc}{\operatorname{soc}}
\newcommand{\PSL}{\operatorname{PSL}}
\newcommand{\GL}{\operatorname{GL}}
\newcommand{\PGL}{\operatorname{PGL}}
\newcommand{\SL}{\operatorname{SL}}
\newcommand{\PG}{\operatorname{PG}}
\newcommand{\rank}{\operatorname{rank}}
\newcommand{\Image}{\operatorname{Im}}
\newcommand{\tr}{\operatorname{tr}}
\newcommand{\Fix}{\operatorname{Fix}}
\newcommand{\PGammaL}{\operatorname{P\Gamma L}}
\newcommand{\PSigmaL}{\operatorname{P\Sigma L}}
\newcommand{\val}{\operatorname{val}}
\newcommand{\PSp}{\operatorname{PSp}}
\newcommand{\Sp}{\operatorname{Sp}}
\newcommand{\PSU}{\operatorname{PSU}}
\newcommand{\POmega}{\operatorname{P\Omega}}
\newcommand{\M}{\operatorname{M}}
\newcommand{\Co}{\operatorname{Co}}
\newcommand{\McL}{\operatorname{McL}}
\newcommand{\HS}{\operatorname{HS}}
\newcommand{\Suz}{\operatorname{Suz}}
\newcommand{\J}{\operatorname{J}}
\newcommand{\Fi}{\operatorname{Fi}}
\newcommand{\Th}{\operatorname{Th}}
\newcommand{\HN}{\operatorname{HN}}
\newcommand{\He}{\operatorname{He}}
\newcommand{\Ly}{\operatorname{Ly}}
\newcommand{\Ru}{\operatorname{Ru}}
\newcommand{\Sz}{\operatorname{Sz}}
\newcommand{\Ree}{\operatorname{Ree}}
\newcommand{\AGL}{\operatorname{AGL}}
\newcommand{\Cos}{\operatorname{Cos}}
\newcommand{\Cay}{\operatorname{Cay}}
\newcommand{\LL}{\operatorname{L}}
\newcommand{\UU}{\operatorname{U}}
\newcommand{\SU}{\operatorname{SU}}
\newcommand{\PGU}{\operatorname{PGU}}
\newcommand{\PGammaU}{\operatorname{P\Gamma U}}
\newcommand{\GU}{\operatorname{GU}}
\newcommand{\FF}{\mathbb{F}}
\newcommand{\ZZ}{\mathbb{Z}}
\newcommand{\lr}{\langle}
\newcommand{\rr}{\rangle}
\def\calc{\mathcal{C}}
\def\cali{\mathcal{I}}





\theoremstyle{plain}

\newtheorem{theorem}{Theorem}
\newtheorem{proposition}{Proposition}[section]
\newtheorem{lemma}[proposition]{Lemma}
\newtheorem{corollary}[proposition]{Corollary}
\newtheorem{thm}[proposition]{Theorem}

\newtheorem{conjecture}{Conjecture}
\newtheorem{conj}{Conjecture}
\newtheorem{problem}{Problem}

\theoremstyle{definition}

\newtheorem{definition}[proposition]{Definition}
\newtheorem{remark}[proposition]{Remark}
\newtheorem{example}[proposition]{Example}

\begin{document}
	Let $\FF_q$ be a finite field and $p=\mathrm{char}(\FF_q)$. Let $G$ be an almost simple primitive permutation group with socle $G_0=\PSU_3(q)$ and stabilizer $H$. In this section we characterize all the intersecting subgroups of $G$. 
	
	We first assume  $G$ is simple, which means $G=G_0$. The maximal subgroups of $G$ are recorded in 'low dimension book', we will say a little bit more about the classes of elements in the Sylow-p subgroup of $G$.
	
	\begin{lemma}\label{Syl_p}
		Let $Q$ be the Sylow-$p$ subgroup of $G$, and let $N=N_G(Q)$ be the normalizer of $Q$. Then the elements of $Q$ can be represented as $Y(b,a)$ and the elements of $N$ can be represented as $(Y(b,a),\tau(\lambda))$ where $a, b,\lambda \in \FF_{q^2}^{\times}$ with the following computation method.
		\begin{enumerate}[\rm(i)]
			\item $a+a^q+bb^q=0$
			\item $Y(b_1,a_1)Y(b_2,a_2)=Y(b_1+b_2,a_1+a_2-b_1^qb_2)$
			\item $Y(b_0,a_0)^{-1}Y(b,a)Y(b_0,a_0)=Y(b,a+b_0^qb-b^qb_0)$
			\item 
			$Y(b,a)^{\tau(\lambda)}=Y(\lambda^{1-2q}b,\lambda^{-(q+1)}a)$
		\end{enumerate}
	\end{lemma}
\begin{proof}
	Let $(V,B)$ be a $3$-dimensional unitary space over $\FF_q^2$ with a basis $\{e_1,f_1,e_2\}$, where $(e_1,f_1)$ is a hyperbolic pair and $e_2$ is a norm-$1$ vector perpendicular to $\lr e_1, f_1\rr$.
	
	By a quick examination at the table we find that the Sylow-$p$ subgroup of $G$ is contained in the stabilizer of an isotropic space, therefore we may assume $Q\le G_{\lr e_1\rr}$.
	
	By calculation, the matrices of $G_{\lr e_1\rr}$ are $A{:}B$, where $B=\{\mathrm{diag}(\lambda,\lambda^{-q},\lambda^{q-1})\mid \lambda \in \FF_{q^2}^{\times} \}$ and 
	\begin{equation*}
		A=\left\{ \overline{\begin{bmatrix}
			1 & a & -b^q\\
			0 & 1 & 0 \\
			0 & b & 1
		\end{bmatrix}} \mid a, b\in \FF_{q^2},  a+a^q+bb^q=0  \right\},
	\end{equation*}
where the line over the matrix means the project image of this matrix in the unitary groups modulo scalars.
We denote the matrix $$\overline{\begin{bmatrix}
		1 & a & -b^q\\
	0 & 1 & 0 \\
	0 & b & 1
\end{bmatrix}}$$ as $Y(b,a)$, a calculation of matrix multiplication justifies Lemma \ref{Syl_p}.
\end{proof}
\begin{lemma}
	Let $Q$ be the Sylow-$p$ subgroup of $G$, if $p=2$ then $Q$ has exponent $4$, if $p\neq 2$, then $Q$ has exponent $p$.  
\end{lemma}
\begin{proof}
	It follows directly from Lemma \ref{Syl_p} (iii).
\end{proof}
\begin{lemma}\label{Sylp_int}
	Two different Sylow-$p$ subgroups of $G$ intersect trivially.
\end{lemma}
\begin{proof}
	Let $g\neq N_G(Q)$, then $Q\neq Q^g$ and $Q\cap Q^g=G_{\lr e, e^g \rr }\cong \GU_2(q)$. Let $x\in Q\cap Q^g$, one may consider the matrix of $x$ with respect to $\{e_1,e_1^g,f\}$ where $f\in \lr e_1,e_1^g \rr$. We get that $x|_{\lr e_1,e_1^g \rr}$ is  unipotent and $x|_{\lr f \rr}$ is trivial. 
	Moreover, since both $e_1$ and $e_1^g$ are eigenvectors of $x$, we conclude that $x|_{\lr e_1, e_1^g\rr} $ is semisimple and hence is trivial. Combining all above facts, we get $x=1$.	
\end{proof}
One may note that $N_G(Q)$ is solvable with a normal Sylow-$p$ subgroup, then we have the following:
\begin{lemma}\label{subgp}
	Let $I$ be a subgroup of $N_G(Q)$, then $I=(I\cap (Q)){:}C$ where $C$ is conjugate to a subgroup of $\lr \tau(\lambda) \rr$.
\end{lemma}
\begin{proof}
	Clearly we have $(I\cap Q)\trianglelefteq I $ is the Sylow-$p$ subgroup of $I$, since $I\le N_G(Q)$ is solvable, let $I_{p'}$ be a Hall-$p'$ subgroup of $I$, we have $I=(I\cap Q){:}I_{p'}$. 
	Since $\lr \tau(\lambda) \rr$  is a Hall-$p'$ subgroup of $N_{G}(Q)$, we conclude that  $I_{p'}$ is conjugate to a subgroup of $\lr \tau(\lambda) \rr$. 
\end{proof}
\begin{lemma}
	Let $x\in Q\backslash Z(Q)$, then $|C_{N_G(Q)}(x)|=q^2$.
\end{lemma}
\begin{proof}
	It follows directly from Lemma \ref{Syl_p} (iii).
\end{proof}
\begin{lemma}\label{conjp}
	The centre of  $Q$ is elementary of order $q$ and $Q/Z(Q)$ is isomorphic to the additive group of $\FF_{q^2}$.
	
	Let $x_1,x_2\in Q\backslash Z(Q)$, $x_1$ and $x_2$ are conjugate in $G$ if and only if their image in 
	$Q/Z(Q)$ are conjugate in $N/Z(Q)$.
\end{lemma}
\begin{proof}
	By Lemma \ref{Syl_p} (iii), we get that $Z(Q)=\{Y(0,a)\mid a+a^q=0 \}$. 
	By Lemma \ref{Syl_p} (ii), one get that $Z(Q)$ is elementary abelian and the first statement holds.
	
	Since two different Sylow-$p$ subgroups of $G$ intersects trivially, $x_1$ and $x_2$ are conjugate in $G$ if and only if they are conjugate in $N_G(Q)=N$.
	Lemma \ref{Syl_p}(iii) implies that if $x\notin Z(Q)$, all elements in  $xZ(Q)$ are conjugate in $Q$, 
	hence the second statement holds.
\end{proof}

\begin{lemma}
	All non-trivial elements in $Z(Q)$ are conjugate. If $q\nequiv -1 \pmod 3$, then all elements in $Q\backslash Z(Q)$ are conjugate, if $q\equiv -1 \pmod 3$, then the elements in $Q\backslash Z(Q)$ forms three $G$ classes.
\end{lemma}
\begin{proof}
	Since $Z(Q)=\{Y(0,a)\mid a+a^q=0 \}$, Lemma \ref{Syl_p}(iv) implies the first statement.
	
	Let $\lambda $ be a generator of $\FF_{q^2}^{\times}$.
	By calculation we get $(2q-1,q^2-1)=(3,q+1)$, hence if $q\nequiv -1 \pmod 3$, then $N/Z(Q)$ is isomorphic to $\AGL_1(q^2)$ and all non-identity elements in $Q/Z(Q)$ are conjugate in $N/Z(Q)$.
	If $q \equiv -1 \pmod 3$ then $N/Z(Q)\cong \FF_{q^2}^{+}{:}\lr \lambda^3\rr $ and the non-identities in $Q/Z(Q)$ forms three $N/Z(Q) $-classes, and the representative for the three classes are $\{Y(b,a)\mid b\in \lr\lambda^3\rr  \}$, $\{Y(b,a)\mid b\in \lambda\lr \lambda^3\rr  \}$ and $\{Y(b,a)\mid b\in \lambda^2\lr \lambda^3\rr  \}$ respectively.
\end{proof}

We then say a little bit more about the semi-simple elements of $G$ in the following Lemma.
\begin{lemma}\label{semisimple}
	All cyclic subgroups of order $q^2-1$ are conjugate in $G$, and all cyclic subgroups of order $q^2-q+1$ are conjugate, any semi-simple element of $G$ are contained in a cyclic subgroup of order $q^2-1$ or in a cyclic subgroup of order $q+1$ or in a cyclic subgroup of order $q^2-q+1$.
\end{lemma}
\begin{proof}
  Let $x\in G$ be a semi-simple element and let $\lambda_1,\lambda_2$ and $\lambda_3$ be three eigenvalues of $x$ in $\overline{\FF_{q^2}}$. 
  If $x$ is diagonalizable over $\FF_{q^2}$, $\lambda_1,\lambda_2,\lambda_3\in \FF_{q^2}$.
  By \textbf{Derangement and prime's thm} $x$ is conjugate to $x^{-\varphi T}$, hence $\{\lambda_1,\lambda_2,\lambda_3\}=\{\lambda_1^{-q},\lambda_2^{-q},\lambda_3^{-q}\}$, if $\lambda_i^{q+1}\neq 1$ for some $i\in[3]$,  then $$\{\lambda_1,\lambda_2,\lambda_3\}=\{\mu, \mu^{-q}, \mu^{q-1} \}$$ for some $\mu $ such that $\mu^{q+1}\neq 1$ and by \textbf{derangement and prime thm} $x$ is $\GU_3(q)$-conjugate to an element of the cyclic subgroup $C$ of order $q^2-1$ where $$C=\{\mathrm{diag}(\mu_0,\mu_0^{-q},\mu_0^{q-1})\mid \mu_0\in \FF_{q^2}^{\times} \}.$$ 
  
  
  If $\lambda_i^{q+1}=1$ for $i \in [3]$, then the order of $x$ divides $q+1$.
  
  Moreover if $x$ is not diagonalizable over $\FF_{q^2}$, then we claim that $\lr x \rr$ acts irreducibly on $V$.
  Assume $U$ is a $2$-dimensional $x$-subspace, then $x|_U\in \GU_2(q)$ and hence the order of $x$ divides $q-1$ or the order of $x$ divides $q+1$, $x$ is diagonalizable over $\FF_{q^2}$ anyway, hence the claim holds.
   Therefore the three eigenvalues of $x$ are $\{\lambda, \lambda^{q^2},\lambda^{q^4} \}$,
   for some $\lambda$ in $\FF_{q^6}\backslash\FF_{q^2}$. Since $x$ is conjugate to $x^{-\varphi T}$,  we have that 
   \begin{equation*}
   	\{\lambda, \lambda^{q^2},\lambda^{q^4} \}=\{\lambda^{-q}, \lambda^{-q^3}, \lambda^{-q^5}  \}
   \end{equation*}
   If $\lambda^{-q}=\lambda$, then $\lambda^{q+1}=1$ and $\lambda \in \FF_{q^2}$, if $\lambda^{-q}=\lambda^{q^2}$, then $\lambda^{q(q+1)}=1$ and hence  $\lambda \in \FF_{q^2}$, therefore we conclude that $\lambda^{-q}=\lambda^{q^4}$ and we have the following 
   \begin{equation*}
   	\begin{aligned}
   	\lambda^{q(q^3+1)}&=1\\
   	\lambda^{1+q^2+q^4}&=1
   \end{aligned}
   \end{equation*}
Therefore, $\lambda^{q^2-q+1}=1$ since the greatest common divisor of $q^4+q^2+1$ and $q^3+1$ is $q^2-q+1$.


Let $x$ and $y$ be two elements of order $q^2-q+1$, then we consider the eigenvalues of $x$ and $y$ and conclude that $x$ has the same rational canonical form with $y^i$ for some integer $i$ and hence $\lr x \rr$ and $\lr y \rr$ are conjugate in $\GU_3(q)$, they are conjugate in $\SU_3(q)$ since their centralizer has full image in $\GU_3(q)/\SU_3(q)$.
Similarly, we consider the eigenvalues of two elements of order $q^2-1$ and get that two cyclic subgroups of order $q^2-1$ are conjugate in $\SU_3(q)$.
\end{proof}

We are now ready to determine the intersecting subgroups contained in $N=N_G(Q)$ when $H$ is of type $\SU_3(q_0)$, they are recorded in the following lemma.

\begin{lemma}
	Let $\FF_{q_0}$ be a subfield of $\FF_q$ such that $[\FF_q:\FF_{q_0}]$ is an odd prime.  Assume $H=\SU_3(q_0)$, then
	\begin{enumerate}[\rm(i)]
		\item If  $[\FF_q:\FF_{q_0}]\neq 3$ or $3\notdivides q-1$, the maximal intersecting subgroup of  $G$ contained in $N$ is $Q{:}\{\tau(\lambda)\mid \lambda\in \FF_{q^2_0} \}$.
		\item If $[\FF_q:\FF_q]=3$ and $q\equiv 1 \pmod 3$, let $K$ be a maximal intersecting subgroup of $G$, then $K=(K\cap Q){:}\{\tau(\lambda)\mid \lambda\in \FF_{q^2_0} \}$, where $K\cap Q\subseteq  \{Y(b,a)\mid b\in \lr \lambda^3 \rr   \}$ and $\lambda$ is generator of $\FF_{q^2}$.
	\end{enumerate}
\begin{proof}
	If $[\FF_q:\FF_{q_0}]\neq 3$, $Q\cap H=\{Y(b,a)\mid a,b\in \FF_{q^2_0}  \}$.
	If $q\equiv -1 \pmod 3$, since $3 \notdivides \frac{q^2-1}{q_0^2-1}$, $\FF_{q^2_0}^\times$ is not contained in the cyclic subgroup of index $3$ of $\FF_{q^2}^{\times}$, we conclude that $Q\cap H$ contains all three $G$-classes of elements in $Q\backslash Z(Q)$.
	If $3\notdivides q+1$, then  all elements in $Q\backslash Z(Q)$ are conjugate and hence the elements in $Q$ are all conjugate to $H\cap Q$.
	 Let  $k\in Q{:}\{\tau(\lambda)\mid \lambda\in \FF_{q_0}\}$, then either $k\in Q$ or $k$ is conjugate to the unique cyclic subgroup of order $q_0^2-1$ in $N$. If $k\in Q$ since $H$ contains all $G$ classes of elements in $Q$, $k$ is conjugate to an element of $H$.
	If $k$ is contained in the unique cyclic subgroup of  order $q_0^2-1$ in $N$, since this cyclic subgroup is in $H$, we conclude $k$ is conjugate to an element in $H$.
	On the other hand, let $K_0$ be an intersecting subgroups contained in $N$, by Lemma \ref{subgp} we have $K_0=(K_0\cap Q){:}K_{0p'}$, $K_{0p'}\subseteq q_0^2-1$ since the semisimple elements in $H$ are of order dividing $q_0^2-1$ or of order dividing $q_0^2-q_0+1$ and $(q_0^2-q_0+1,q^2-1)=(q_0+1,3)$. Therefore we get $K_0\le Q{:}\{\tau(\lambda)\mid \lambda\in \FF_{q^2_0}\}.$
	
	If $[\FF_q{:}\FF_{q_0}]=3$, and let $K$ be an intersecting subgroup of $G$, similarly by Lemma \ref{subgp} we have that $K=(K\cap Q){:}K_{p'}$. By Lemma \ref{semisimple},
    $|K_{p'}|$ divides $q_0^2-1$ or $|K_{p'}|$ divides $q_0^2-q_0+1.$ 
    The later case is not possible since $q_0^2-q_0+1$ divides $q+1$ and the elements in  $N$ or order $q+1$ has two same eigenvalues but the $3$-eigenvalues of  elements in $H$ are all different.
    Since $3\divides\frac{q^2-1}{q_0^2-1}$, $\FF_{q_0^2}^{\times}$ is contained in the index $3$ subgroup of $\FF_q$. Therefore, if $q\equiv 1 \pmod 3$, $H\cap Q$ contains only $1$ $G$-classes of elements in $Q\backslash Z(Q)$. We conclude $K\cap Q$ contains only $1$ $G$-classes of elements in $Q\backslash Z(Q)$. 
\end{proof}
\end{lemma} 


		
		

\end{document}