\documentclass[12pt,reqno]{amsart}
\usepackage{enumerate}
\usepackage{mathrsfs,hyperref}
\usepackage{amsfonts}
\usepackage{epsfig,amsfonts,amsmath}
\usepackage{graphicx}
\usepackage{mathabx}
\usepackage{amsthm}
% \usepackage[doublespacing]{setspace}
\usepackage{textcomp}
\usepackage{amsmath,amssymb}
\usepackage{fancyhdr}
% \usepackage{lastpage}
\usepackage{booktabs}
\usepackage{makecell}
\usepackage{tabularx}
\usepackage{mathrsfs}
\usepackage{ragged2e}
\usepackage{bm}
\usepackage{float}
\usepackage{color}
\usepackage{dynkin-diagrams}
\usepackage{indentfirst}
\usepackage{blkarray}
% \usepackage[square]{natbib}

\usepackage{geometry}

\geometry{a4paper,left=3cm,right=2.5cm,top=2.5cm,bottom=2cm}

%\usepackage{setspace}

%\renewcommand{\baselinestretch}{1.5}%行间距1.5倍




% \renewcommand{\contentsname}{\fontsize{18pt}{\baselineskip}\selectfont \textbf{Contents}}%Contents 18pt





\title[On intersecting maximal subgroups]{On intersecting maximal subgroups}




\date{\today}

\newcommand{\id}{\operatorname{id}}
\newcommand{\Aut}{\operatorname{Aut}}
\newcommand{\Inn}{\operatorname{Inn}}
\newcommand{\Out}{\operatorname{Out}}
\newcommand{\soc}{\operatorname{soc}}
\newcommand{\PSL}{\operatorname{PSL}}
\newcommand{\GL}{\operatorname{GL}}
\newcommand{\PGL}{\operatorname{PGL}}
\newcommand{\SL}{\operatorname{SL}}
\newcommand{\PG}{\operatorname{PG}}
\newcommand{\rank}{\operatorname{rank}}
\newcommand{\Image}{\operatorname{Im}}
\newcommand{\tr}{\operatorname{tr}}
\newcommand{\Fix}{\operatorname{Fix}}
\newcommand{\PGammaL}{\operatorname{P\Gamma L}}
\newcommand{\PSigmaL}{\operatorname{P\Sigma L}}
\newcommand{\val}{\operatorname{val}}
\newcommand{\PSp}{\operatorname{PSp}}
\newcommand{\Sp}{\operatorname{Sp}}
\newcommand{\PSU}{\operatorname{PSU}}
\newcommand{\POmega}{\operatorname{P\Omega}}
\newcommand{\M}{\operatorname{M}}
\newcommand{\Co}{\operatorname{Co}}
\newcommand{\McL}{\operatorname{McL}}
\newcommand{\HS}{\operatorname{HS}}
\newcommand{\Suz}{\operatorname{Suz}}
\newcommand{\J}{\operatorname{J}}
\newcommand{\Fi}{\operatorname{Fi}}
\newcommand{\Th}{\operatorname{Th}}
\newcommand{\HN}{\operatorname{HN}}
\newcommand{\He}{\operatorname{He}}
\newcommand{\Ly}{\operatorname{Ly}}
\newcommand{\Ru}{\operatorname{Ru}}
\newcommand{\Sz}{\operatorname{Sz}}
\newcommand{\Ree}{\operatorname{Ree}}
\newcommand{\AGL}{\operatorname{AGL}}
\newcommand{\Cos}{\operatorname{Cos}}
\newcommand{\Cay}{\operatorname{Cay}}
\newcommand{\LL}{\operatorname{L}}
\newcommand{\UU}{\operatorname{U}}
\newcommand{\SU}{\operatorname{SU}}
\newcommand{\PGU}{\operatorname{PGU}}
\newcommand{\PGammaU}{\operatorname{P\Gamma U}}
\newcommand{\GU}{\operatorname{GU}}
\newcommand{\FF}{\mathbb{F}}
\newcommand{\ZZ}{\mathbb{Z}}
\newcommand{\lr}{\langle}
\newcommand{\rr}{\rangle}
\def\calc{\mathcal{C}}
\def\cali{\mathcal{I}}





\theoremstyle{plain}

\newtheorem{theorem}{Theorem}
\newtheorem{proposition}{Proposition}[section]
\newtheorem{lemma}[proposition]{Lemma}
\newtheorem{corollary}[proposition]{Corollary}
\newtheorem{thm}[proposition]{Theorem}

\newtheorem{conjecture}{Conjecture}
\newtheorem{conj}{Conjecture}
\newtheorem{problem}{Problem}

\theoremstyle{definition}

\newtheorem{definition}[proposition]{Definition}
\newtheorem{remark}[proposition]{Remark}
\newtheorem{example}[proposition]{Example}

\begin{document}
	Let $G$ be an almost simple primitive permutation group with socle $G_0\cong \PSU_3(q)$, let $H$ be a stabilizer of $G$ and let $K$ be a maximal intersecting subgroup such that $|K|\ge |H|$.
	We first assume $G$ is simple, and $G$ does not have a maximal subgroup of Aschbaher class $\mathcal{I}$. Then the order of maximal subgroups of $G$ is ordered as the following:
	\begin{equation}\label{order}
		\begin{aligned}
		&|E_q^{1+2}{:}(q^2-1)/(q+1,3)|>|\GU_2(q)/(q+1,3)|>|\mathrm{SO}_3(q)|>(|\PSU_3(q^{1/3})|\, \text{if $q$ is a cube})\\
		&>|(q+1)^2/(q+1,3).S_3|>|\frac{q^2-q+1}{(q+1,3)}{:}3|>(|\PSU_3(q^{1/r})|\; \text{if $q=q_0^r$ for some odd prime $r$}) \\
		&>(|3^2{:}Q_8.\frac{(q+1,9)}{3}|\; \text{if $p=q\equiv 2\pmod 3$, $q\ge 11$})
	\end{aligned}
	\end{equation}
 \textbf{We omit the calculation part here}.
  Let $Q$ be a Sylow-$p$ subgroup of $G$, we then state a Lemma about subgroups of $Q$.
  \begin{lemma}\label{unipotent}
  	If $S\le Q$ such that all elements in $S\backslash 1$ are conjugate to $Z(Q)$ or all elements in $S\backslash 1$ are not conjugate to $Z(Q)$ in $N_G(Q)\cong Q{:}\frac{q^2-1}{(3,q+1)}$, then $|S|\le q$.
  \end{lemma}
  \begin{proof}
  	Assume all elements in $S$ are conjugate to $Z(Q)$ then $S\le Z(Q)$ and the result holds immediately. 
  	Assume all elements in $S$ are conjugate to elements in $Q\backslash Z(Q)$, then $S\cap Z(Q)=1$ and hence for any $s_1,s_2\in S$ we have 
  	 $$[s_1,s_2]\le S\cap Q'=S\cap Z(Q)=1,$$
  	 Therefore, we have that $S$ is abelian and $S\cap Z(Q)=1$, and hence $|S|\le q$ by (iii) of basic facts \textbf{Lemma}.
  \end{proof}

We state about the Sylow-$2$ and Sylow-$3$ subgroup of $G$ which will be useful later.
\begin{lemma}\label{Syl-2}
	\begin{enumerate}[\rm(i)]
		\item If $q\equiv 1\pmod 4$, then the Sylow-$2$ subgroup of $G$ is isomorphic to $\mathrm{SD}_{4(q-1)_2}$, if $J$ is a $2$-group of $G$ of exponent $4$, then $|J|\le 8.$
		\item If $q\equiv -1 \pmod 4$, then the Sylow-$2$ subgroup of $G$ is isomorphic to $\lr a_1,a_2, b\mid a_1^t=a_2^t=1,a_1a_2=a_2a_1,a_1^b=a_2,a_2^b=a_1 \rr$, where $t=(q+1)_2$ If $J$ is a $2$-group of $G$ of exponent $4$, then $|J|\le 16.$
		\item If $q$ is odd and $J$ is a subgroup of $G$ and $J$ is a $2$-group of exponent $2$, then  $J\cong 2$ or $J\cong 2^2.$
	\end{enumerate}
\end{lemma}
  \begin{proof}
  	Let $G_2$ be the Sylow-$2$ subgroup of $G$. 
  	
  	If $q\equiv 1 \pmod 4$, then $|G_2|=2(q-1)_2$. Let $\mu$ generate the $2$-group of $\FF_{q^2}^{\times}$, we take a basis $\{e_1,f_1,e_2\}$ of a dimension $3$ unitary space over $\FF_{q^2}$ such that $(e_1, f_1)$ is a hyperbolic pair and $e_2$ is a norm $1$ vector in $\lr e_1, f_1 \rr^{\perp}$. Let $a= \mathrm{diag}(\mu, \mu^{-q}, \mu^{q-1}), b=\begin{bmatrix}
  		0 & 1 & 0\\
  		1 & 0 & 0\\
  		0 & 0 & -1
  	\end{bmatrix}$ and $L=\lr a, b\rr$, then $L$ is a subgroup of $\SL_3(q^2)$ of order $4(q-1)_2$, and $L$ contains all unitary matrices with respect to the basis $\{e_1, f_1,e_2\},$ hence $L\cong \bar{L}=L/Z(\SU_3(q))$ is a Sylow-$2$ subgroup of $G$.
   Since $\lambda^{q-1}=-1,$ we have $\lambda^{-q}=-\lambda^{-1}=\lambda^{(q-1)_2-1},$ and hence $\bar{L}\cong \mathrm{SD}_{4(q-1)_2}.$ Let $M$ be a subgroup of $L$, if $M$ is of exponent $4$, then $|M\cap \lr a \rr| \le 4$ and $|M|\le 8$, if $M$ is of exponent $2$, then $|M\cap \lr a \rr|\le 2$ and hence $|M|\le 4$.
   
   If $q\equiv 3\pmod 4$, then $|G_2|=2(q+1)_2^2$ and hence $G_2$ is contained in a maximal $C_2$ subgroup of $G$ isomorphic to $(q+1)^2/(3,q+1){:}S_3,$ and we conclude that the first statement of $(ii)$ holds.  
   Let $G_2=\lr a_1,a_2,b\mid a_1^t=a_2^t=b^2=1,a_1a_2=a_2a_1,a_1^b=a_2,a_2^b=a_1 \rr$, where $t=(q+1)_2$. Let $M$ be a subgroup of $G_2$, if $M$ is of exponent $4$, then $M\cap \lr a_1, a_2 \rr \le \lr a_1^{t/4},a_2^{t/4}\rr$. Let $g\in M\backslash \lr a_1,a_2\rr$, then $M=\lr M\cap \lr a_1,a_2\rr, g  \rr$, we write $g=a_1^ma_2^nb$. Since $g^4=1$ and $g^2=(a_1a_2)^{m+n}=1$, we conclude $m+n$ is $t$ or $t/2$. We note that  $a_1^{-m}ga_1^m=a_2^{m+n}b$, hence 
   \begin{equation*}
   	M^g=\lr  M\cap \lr a_1^{t/4},a_2^{t/4} \rr, a_2^{m+n}b \rr \le \lr a_1^{t/4}, a_2^{t/4},a_2^{m+n}b\rr =\lr a_1^{t/4},a_2^{t/4},b \rr,
   \end{equation*}
   \textbf{We then use Magma to find all subgroup of exponent $2$ and $4$ of $\lr a_1^{t/4}, a_2^{t/4}, t\rr$}, we conclude that $\rm(ii)$ and $\rm(iii)$ hold. 
  \end{proof}

\begin{lemma}\label{Syl-3}
	\begin{enumerate}[\rm(i)]
		\item If $q\equiv -1\pmod 3$, then the Sylow-$3$ subgroup of $G$ is isomorphic to $\lr a_1,a_2,b\mid a_1^t=a_2^t=b^3=a_1a_2=a_2a_1=1,a_1^b=a_1^{-1}a_2,a_2^b=a_1^{-1}  \rr.$
		If $J$ is a $3$-group of exponent $3$, then $|J|\le 3(q+1,9)$, moreover if $|J|=27$, then $J$ is non-abelian.  
		\item 	If $q\equiv 1 \pmod 3$, then the Sylow-$3$ subgroup of $G$ is cyclic.
	\end{enumerate}
	

\end{lemma}
  \begin{proof}
   Let $G_3$ be the Sylow-$3$ subgroup of $G$. 
   
   If $q\equiv -1 \pmod 3$, then one may conclude that $|G_3|=3(q+1)_3^2$, and hence $G_3$ is contained in a maximal subgroup of $G$ isomorphic to $(q+1)^3/(3,q+1){:}S_3.$
   We take an orthogonal normal basis $\{e_1,e_2,e_3\}$ of a $3$-dimensional unitary basis, and let $\mu$ be a generator of the cyclic subgroup of order $(q+1)_3$ of $\FF_{q^2}^{\times}.$ 
   Let $a_1=\overline{\mathrm{diag}(\mu, 1, \mu^{-1})}$, $a_2=\overline{\mathrm{diag}( 1,\mu, \mu^{-1})}$ and $b=\bar{M}$ with respect to $\{e_1,e_2,e_3\}$, where $M$ is a permutation matrix of $(1,2,3)$, and the overline denotes the matrices modulo the center of $\SU_3(q)$. 
   Let $L=\lr a_1,a_2,b\rr$, then $L\le G$ and $|L|=|G_3|$, we check the generating relations of $L$ and conclude the first statement holds.
    
   Let $J$ be a subgroup of $L$, if the exponent of $J$ is $3$, then $J\cap \lr a_1,a_2\rr \le \Omega_1(\lr a_1, a_2 \rr)$, hence $|J\cap \lr a_1,a_2 \rr|\le (q+1,9)$ and $|J|\le 3(q+1,9)$. 
   If $|J|=27$, then $9$ divides $q+1$ and $J\cap \lr a_1,a_2\rr =\lr a_1^{t/3},(a_1a_2)^{t/9}\rr,$ let $g\in J\backslash(\lr a_1, a_2 \rr)$, we write $g=a_1^ma_2^nb$ and we conclude $J$ is not abelian since $g$ does not commute with $(a_1a_2)^{t/9}$.  Therefore, we have already shown the statements of $(i)$.
   
   If $q\equiv 1 \pmod 3$, then $|G_3|=(q-1)_3$ and hence $G_3$ is cyclic since there is a cyclic subgroup of order $q-1$ in $G$.
  \end{proof}
  
  \begin{lemma}\label{Syl-r}
  	If $r$ is a prime and $r\notin\{2,3,p \},$ then the Sylow-$r$ subgroup is either cyclic or a direct product of two cyclic groups.
  \end{lemma}
  \begin{proof}
  	It is clear that $r$ divides exactly one of $q-1$, $q+1$ and $\frac{q^2-q+1}{(3,q+1)}.$ If $r$ divides $q-1$ or $\frac{q^2-q+1}{(3,q+1)}$, then the Sylow-$r$ subgroup of $G$ is cyclic since $G$ contains cyclic subgroups of order $q-1$ and $\frac{q^2-q+1}{(3,q+1)}.$
  	If $r$ divides $q+1$, then the Sylow-$r$ subgroup of $G$ is contained in a subgroup isomorphic to $(q+1)^2/(3,q+1)$, and hence it is isomorphic to $(q+1)_r^2.$
  \end{proof}
  Since $|K|\ge |H|$, we conclude $H$ is not a parabolic subgroup by (\ref{order}).
  Assume $H$ is a stabilizer of a two dimensional subspace then $H\cong \GU_2(q)/(q+1,3)$, hence by (\ref{order}) we conclude $K$ is contained in a maximal parabolic subgroup.
  Up to a conjugation, we may assume $K\le N_G(Q)$ and $K=(K\cap Q){:}K_{p'}$.
  Since all unipotent element in $H$ are conjugate, we conclude that $|K\cap Q|\le q$ by Lemma \ref{unipotent} and $|K|\le q\frac{q^2-1}{(3,q+1)}<|H|$, a contradiction.
  
  Assume $H\cong \mathrm{SO}_3(q)$ where $q$ is odd and $q\ge 7$, then by (\ref{order}) we have that $K$ is contained in a maximal parabolic subgroup or $K$ is contained in a $2$-subspace stabilizer.
  If $K$ is contained in a maximal parabolic subgroup, up to a conjugation we assume $K\le N_G(Q)$ and $K=(K\cap Q){:}K_{p'}$. 
  Since $H$ contains only $1$ class of unipotent elements, hence $|K\cap Q|\le q$ by Lemma \ref{unipotent}. We also have $|K_{p'}|\le q+1$ since the order of semisimple elements in $H$ are a divisor of $q-1$ or a divisor of $q+1$. 
  Therefore, we conclude that $|K|=|K\cap Q||K_{p'}|\le q(q+1)< |H|$, a contradiction.
  If $K$ is contained in a $2$-subspace stabilizer and $\mathrm{Char}(q)\neq 2$, then by \textbf{Lemma} of basic fact, we conclude that the unipotent elements in $H$ and the $2$-subspace stabilizer are not conjugate.
  Therefore, we conclude that $|K|$ is coprime to $p$ and hence $K$ is contained in a maximal subgroup of the $2$-subspace stabilizer of Aschbacher class $\mathcal{C}_2$, $\mathcal{C}_3$ or $\mathcal{I}$, hence $|K|<|H|$. 
  
  Assume $H$ is a maximal $\mathcal{C}_2$ subgroup and $H\cong (q+1)^2/(3,q+1){:}S_3$.
  %If $p=\mathrm{Char}(q)\neq 2,3$, we have that all elements in $H$ are semisimple. 
  %If $K$ is contained in a maximal parabolic subgroup, then $K\le K_{p'}$ and hence $|K|<|H|$. If $K$ is contained in a $2$-subspace stabilizer, then $K$ is contained in a maximal subgroup of Aschbacher class $\mathcal{C}_2$, $\mathcal{C}_3$ or $\mathcal{I}$ of the $2$-subspace stabilizer, and hence $|K|<|H|$. 
  %Similarly, if $K\lesssim \mathrm{SO}_3(q)$ then we conclude $|K|<|H|$. If $K$ is a subgroup of $\PSU_3(q^{1/3}).(\frac{q+1}{q^{1/3}+1})$.
  %If $\mathrm{Char}(q)=2,3$ we have that 
  Since the order of unipotent elements in $H$ are $2$ or $3$, we conclude that $H$ contains at most $1$ class of unipotent elements. 
  We also note that the order of semisimple elements in $H$ are divisors of $2(q+1)$ or $3$.
  If $K$ is contained in a maximal parabolic subgroup, then $|K_p|\le q$ by Lemma (\ref{unipotent}) and $|K_{p'}|\le 2(q+1)$ since $\mathrm{Spec}(K)\subseteq \mathrm{Spec}(H)$, hence $|K|=|K_{p}||K_{p'}|<|H|$.
  If $K$ is contained in a $2$-subspace stabilizer $K_1\cong \GU_2(q)/(3,q+1)$, we conclude that $K$ does not contain a subgroup isomorphic to $\SU_2(q_0)$ or $q_0\in \{3,4,5\}$ since $\SU_2(q_0)$ contains a semisimple element of order $q_0-1$ and $(q_0-1,2(q+1))$ divides $4$. 
  On the other hand, since $|K|\ge |H|$ we conclude $K$ contains the unique subgroup of $K_1$ isomorphic to $\SU_2(q)$, or $K$ is a maximal subfield subgroup of $K_1$, or $K\lesssim \SL_2(5)\circ \frac{q+1}{(3,q+1)}$ and $q\le 9$, or $K\lesssim \GL_2(3)\circ \frac{q+1}{(3,q+1)}$ and $q\le 7$. 
  Therefore, we conclude $q\le 9$ and leave the remaining discussion to Magma. 
  If $K$ is contained in a maximal subgroup $K_1$ such that $K_1\cong \mathrm{SO}_3(q)$, then since $|K|\ge |H|$ we conclude $K=K_1\cong \mathrm{SO}_3(q)$, then there is a semisimple element in $H$ of order $q-1$ since $q-1\in \mathrm{Spec}(K)\subseteq \mathrm{Spec}(H)$, therefore, we conclude that $q\in\{3,4,5\}$ and leave the remaining discussion to Magma. 
  If $q$ is a cube and $K$ is a subgroup of $\PSU_3(q^{1/3}).(q+1,3)$. Since $|K|\ge |H|$, we conclude that either $K=K_1$ or $K$ is contained in a maximal subgroup of $K_1$ of Aschbacher class $\mathcal{I}$. 
  Moreover, we conclude $K$ is not $K_1$ since $K_1$ contains at least $2$ classes of unipotent elements. 
  As a result, we conclude $K$ is contained in a maximal subgroup of $K_1$ in Aschbacher class $\mathcal{I}$ and  $K\lesssim \LL_2(7)$, or $K \lesssim A_6$ and $q^{1/3}\equiv 11, 14  \pmod {15}$, or $q^{1/3}=5$ and $K\lesssim  A_6.2$ or $K\lesssim A_7$, but none of the above cases are possible since $|K|\ge |H|$.
  By (\ref{order}), we conclude that no intersecting subgroup of $G$ satisfy $|K|\ge |H|$.
  
  Assume $H$ is a maximal $\mathcal{C}_3$ subgroup and $H\cong \frac{q^2-q+1}{(3,q+1)}{:}3$, then by (\ref{order}) we conclude that $K$ is contained in a maximal subgroup isomorphic to one of $E_q^{1+2}{:}\frac{q^2-1}{(3,q+1)},\GU_2(q)/(3,q+1), \mathrm{SO}_3(q), \PSU_3(q^{1/3})$ and $(q+1)^2/(3,q+1){:}S_3$.
   We observe that the order of the semisimple elements in $H$ are a divisor of $q^2-q+1$ or $3$, we also observe that if $H$ contains unipotent elements, then the unipotent elements in $H$ are conjugate to the Sylow-$3$ subgroup $H_3$ of  $H$ and $H_3\cong 3$.
  Since $(q^2-q+1,q^2-1)=(3,q+1)$ and $q^2-q+1$ is always odd, we conclude if $K$ is contained in a maximal subgroup isomorphic to one of $\mathrm{E}_q^{1+2}{:}\frac{q^2-1}{(3,q+1)}$, $\GU_2(q)/(3,q+1)$, $\mathrm{SO}_3(q)$, $\PSU_3(q^{1/3}).(3,q+1)$ and $(q+1)^2/(q+1,3){:}S_3$, then the order of the semisimple elements in $K$ is $3$. 
  Therefore, we conclude in above cases, either all elements in $K$ is semisimple of order $3$ or all elements in $K$ is unipotent of order $3$ and hence $K$ is a $3$-group.
  On the other hand, we examine the Sylow-$3$ subgroup of $G$. If $q\equiv 0 \pmod 3$, then $q=3^t$ for some integer $t$ and $K\le Q$, hence $|K|\le q$ since all elements in $K$ are conjugate to a cyclic subgroup and we apply Lemma (\ref{unipotent}). If $q\equiv -1 \pmod 3$, then we conclude $K$ is contained in the cyclic subgroup of order $\frac{q^2-1}{(3,q+1)}$ and $K\cong 3$. If $q\equiv 1\pmod 3$, then $K\lesssim (q+1)^2/(3,q+1)$ and hence $K\cong 3^2$.
  In all these case we have $|K|<|H|$.
  
  If $H$ is isomorphic to $\PSU_3(q_0).(\frac{q+1}{q_0+1},3)$, where $q=q_0^r$ and $r$ is an odd prime, then the order of  semisimple elements in $H$ are a divisor of $\frac{q_0^2-1}{(3,q_0+1)}$ or a divisor of $\frac{q_0^2-q_0+1}{(3,q_0+1)}.$   
  We observe that $(q_0^2-1,q+1)=q_0+1$, we also observe that $q_0^2-q_0+1$ divides $q+1$ if $q=q_0^3$ while $(q_0^2-q_0+1,q^2-1)=(q_0^2-q_0+1,q+1)=(3,q_0+1)$ if $q=q_0^r$ and $r\neq 3$.
  
  \textbf{If $K$ is a subgroup of the maximal parabolic subgroup of $G$, then we discussed it in detail.}
  If $H$ is isomorphic to $\PSU_3(q_0).(\frac{q+1}{q_0+1},3)$ and $K$ is contained in a maximal subgroup $K_1$ in Aschbacher class $\mathcal{C}_2$, then $K_1\cong (q+1)^2/(3,q+1){:}S_3$. Assume $K_1=M{:}H$, where $M\cong (q+1)^2/(3,q+1)$ and $H\cong S_3$. 
  We note that each element in $K\cap M$ is semisimple, hence $K\cap M\lesssim (q_0+1)^2/(3,q_0+1)$ or $K\cap M \lesssim (q_0^2-q_0+1)^2/(3,q_0+1)$ and $q=q_0^3$ since $\mathrm{Spec}(K)\subseteq \mathrm{Spec}(H)$. Therefore, we conclude that $|K|=|K\cap K_1||K/K\cap K_1|<|H|,$ a contradiction. 
  By a similar argument, we conclude that $K$ is not contained in a maximal subgroup of Aschbacher class $\mathcal{C}_3$.
  
  If $H$ is isomorphic to $\PSU_3(q_0).(\frac{q+1}{q_0+1},3)$ and $K$ is contained in a $2$ subspace stabilizer $K_1\cong \GU_2(q)/(3,q+1)$, then $K\neq K_1$ since $\frac{q^2-1}{(3,q+1)}\in \mathrm{Spec}(K_1)\backslash \mathrm{Spec}(H)$.
  Let $K_2$ be the normal subgroup of $K_1$ isomorphic to $\SU_2(q)$, by the classification of the subgroup of $\LL_2(q)$ of Dickson we conclude that either $K\cap K_1$ is isomorphic to one of $\SL_2(q_1)$, $\GL_2(q_1)$, $\GL_2(3)$, $\SL_2(3)$ and $\SL_2(5)$ or 
  $K$ is contained in a maximal subgroup of $K_1$ isomorphic to one of $q{:}\frac{q^2-1}{(3,q+1)}$, $(q+1)\wr 2$ and $\frac{q^2-1}{(3,q+1)}{:}2$.
  If $K\le q{:}\frac{q^2-1}{(3,q+1)}$, then we conclude that $K$ is contained in a maximal parabolic subgroup of $G$ and we have analyzed it. 
  If $K$ is contained in a maximal subgroup of $K_1$ isomorphic to $(q+1)\wr 2$, then $K$ is contained in a maximal $\mathcal{C}_2$ subgroup of $G$ and we already conclude this is not possible.
  If $K\le M\le K_1$, where $M_1$ is a maximal subgroup of $K_1$ isomorphic to $q^2-1/(3,q+1){:}2$, we write $M$ as $M_0{:}t$ where $M_0\cong (q+1)^2/(3,q+1)$. 
  We note that $K\cap M_0\lesssim q_0^2-q_0+1$ or $K\cap M_0\lesssim q_0+1$ since $\mathrm{Spec}(K)\subseteq \mathrm{Spec}(H)$, and hence $|K|<|H|$, a contradiction.
  If $K\cap K_2\cong \SL_2(q_1)$, since $\{q_1-1,q_1+1\}\in \mathrm{Spec}(K)\subseteq \mathrm{Spec}(H)$, we conclude $K$ contains semisimple elements of order $q_1-1$ and $q_1+1$. 
  Since $(q_1-1,q_0^2-q_0+1)=1$, we conclude $q_1-1$ divides $q_0^2-1$, and hence $(q_1-1)$ divides $(q_0^2-1,q-1)=(q_0-1)$ and $\FF_{q_1}\subseteq \FF_{q_0}$, and then $|K|<|H|$, a contradiction.
  If $K\cap K_2$ is a finite group, since $K/(K\cap K_2)$ is cyclic and $(|K/(K\cap K_2)|,p)=1$, we conclude that $H$ contains a semisimple element of order $|K/(K\cap K_2)|$ and $|K/K\cap K_2|$ is a divisor of $q_0^2-q_0+1$ or $q_0^2-1$.
  Since $|K|\ge|H|$, we conclude that $|\SL_2(5)|\ge \frac{|\UU_3(q_0)|}{q_0^2-1}$ and $q_0=2$, we further conclude that  $K\cap K_2\cong A_4$ and $|K|<|H|$, a contradiction.
  
  By a similar argument, we conclude that if $H$ is isomorphic to $\PSU_3(q_0).(\frac{q+1}{q_0+1},3)$ then $K$ is not contained in a subgroup isomorphic to $\mathrm{SO}_3(q)$.
  
  If $H$ is isomorphic to $\PSU_3(q_0).(\frac{q+1}{q_0+1},3)$ and $K$ is contained in another subfield subgroup $K_1$ such that $K_1\cong \PSU_3(q_1).(\frac{q+1}{q_1+1},3)$.
  We conclude that $q_1>q_0$ since $|K|\ge |H|$. 
   Let $M\le K_1$, then $M$ is \textcolor{red}{isomorphic to $\PSU_3(q_2).(\frac{q_1+1}{q_2+1},3)$, or $M$ is contained in a maximal subgroup of $G$ in Aschbacher class $\mathcal{C}_1,\mathcal{C}_2,\mathcal{C}_3$, or $M$ is contained in a maximal subgroup of $G$ of type $\mathrm{SO}_3(q)$, or $H$ is one of $\LL_2(7)$, $A_6$, $A_6.2$ and $A_7$.  } \textbf{Using a Lemma.}
   We note that if $K\cong \PSU_3(q_2).(\frac{q_1+1}{q_2+1},3)$, then $q_2\ge q_0$ since $|K|\ge |H|$, but then $\frac{q_2^2-1}{(3,q_2+1)}\in \mathrm{Spec}(K)\backslash\mathrm{Spec}(H)$, a contradiction.
   If $K$ is one of $\LL_2(7), A_6, A_6.2$ or $A_7$, then $|K|\ge |H|$ implies that $q_0=2$, and then $K$ is a $2,3$ group since $\mathrm{Spec}(K)\subseteq \mathrm{Spec}(H)$, a contradiction. 
   We have discussed the remaining possible choices of $K$ in the previous paragraph.
   
   Assume $H$ is a maximal subgroup of $G$ in Aschbacher class $\mathcal{C}_6$, then we have $H\cong 3^2{:}Q_8.(\frac{(q+1,9)}{3})$ and $q=p\equiv -1 \pmod 3$ and $q\ge 11$. 
   We note that $\mathrm{Spec}(K)\subseteq \mathrm{Spec}(H)\subseteq\{1,2,3,4,6\}$, then $K$ is a $2,3$ group. Let $K_2$ \textbf{may abuse the notation} be the Sylow-$2$ subgroup of $K$, then $K_2$ is of exponent $4$ and $|K_2|\le 16$ by Lemma (\ref{Syl-2}).
   Let $K_3$ and $H_3$ be the Sylow-$3$ subgroup of $H$ and $K$ respectively, then we have $|K_3|\ge |H_3|$ since $|K|\ge |H|$. Since $K_3$ is of exponent $3$, we conclude that $|K_3|=|H_3|$ by Lemma (\ref{Syl-3}).
   We examine all the maximal subgroups of $G$ and conclude that a maximal subgroup containing $K_3$ is in Aschbacher class $\mathcal{C}_2$ or $\mathcal{C}_6$. 
   If $K\le K_1$ such that $K_1$ is a maximal $C_2$ subgroup and $K_1\cong (q+1)^2/3{:}S_3$. Then $K$ is contained in the Hall-${2,3}$ subgroup of $K_1.$ 
   Let $K_1=M{:}N$ where $M\cong (q+1)^2/3$ and $N\cong S_3$, then $K\cap M=\Omega_1(M_3)$ and $3$ divides $|K/K\cap M|$ since $|K_3|=|H_3|$, and hence $|K\cap M|_2$ is of exponent $2$ since $12\notin \mathrm{Spec}(K).$
   Since $|K_2|\ge 8$, we conclude that $|K\cap M|_2=\Omega_1(M_2)\cong 2^2$ and $|K/K\cap M|$ is even and $K/K\cap M\cong S_3.$ 
   We conclude that $\{m^6=1 \mid m\in M\}\subseteq K$, we assume $M$ consists of diagonal matrices and $N$ consists of permutation matrices and let $k_0\in K$ and $k_0=\overline{\mathrm{diag}(-1,\omega,-\omega^2)}$, where $\omega$ is a primitive $3$-rd root of unity of $\FF_p^2.$  
   We conclude that there exists $k\in K$ such that $k=k_mP$, where $k_m\in M$ and $P$ is the permutation matrix for $(1,2)$, since $K/K\cap M\cong S_3.$ 
   We conclude $k^2\in M$ and the order of $k^2$ is $1,2$ or $3$ since $\mathrm{Spec}(K)\subseteq \mathrm{Spec}(H)=\{1,2,3,4,6\},$ and hence $1\neq k^2\in \lr k_0k_0^P \rr$ or $\lr k^2 \rr \cap \lr k_0k_0^P \rr=(1)$.
   If $\lr k^2 \rr \cap  \lr k_0k_0^P \rr=(1)$, then $(k_0k^{-1})^2=k_0k_0^Pk^{-2}$ and $(k_0k^{-1})^2$ is of order $6$ since $k_0k_0^P$ commutes with $k^{-2}$, and $\lr k^{-2} \rr \cap \lr k_0k_0^p \rr$ is trivial, therefore we have $k_0k^{-1}\in K$ and $k_0k^{-1}$ is of order $12$, a contradiction.
   If $1\neq k^2\in k_0k_0^P,$ then we have $k^2=(k_0k_0^P)^i$ for $i\in [2,3,4]$, then we conclude that $(k_0^{1+i}k^{-1})^2=k_0k_0^P$ and hence $k_0^{1+i}k^{-1}$ is an element of order $12$ in $K$, a contradiction. 
   
   Consequently, $K$ is contained in a maximal subgroup of $G$ in Aschbacher class $\mathcal{C}_6$, and we conclude $K$ is maximal since $|K|\ge |H|.$
   We conclude that $K$ is also a maximal $\mathcal{C}_6$ subgroup and $K$ is conjugate to $H$ in $G.\delta$, we also conclude that $9$ divides $q+1$ since $K$ is not conjugate to $H$. 
   We note that $K\cong 3^2:Q_8.3,$ and $\mathrm{Spec}(K)\subseteq\{1,2,3,4,6\}$(which means the order of elements in $K$).  
   Hence we conclude that every element in $K$ is conjugate to $H$ in $G.\delta.$ 
   Let $k\in K$ and let $t$ be the order of $k$, if $t$ is $2,3$ or $6.$ 
   Then since $t$ divides $q+1$, we conclude by (\textbf{we add some reference in derangements and prime}.) that $t$ is conjugate to 
   $\overline{\mathrm{diag}(\lambda_1,\lambda_2,\lambda_3)}$ in $G. \lr \delta \rr,$ where $\lambda_1\lambda_2\lambda_3=1.$
   Hence $C_{G.\lr \delta \rr}(t)=3C_{G}(t)$ by direct calculation.
   If $t=4$, then since $t$ divides $q^2-1$, we conclude that $k$ is conjugate to a diagonal matrix in $G$ if $4$ divides $q+1$, and $k$ is conjugate to \textbf{Think about how to state it smoothly.}  
          
   Then we assume $H$ contains a maximal subgroup in Aschbacher class $\mathcal{I}$, then it suffices to conclude the cases when $K$ is contained in a maximal $\mathcal{I}$ subgroup and the cases when $H$ is a maximal $\mathcal{I}$ subgroup.
   
   If $K$ is contained in a maximal subgroup isomorphic to $\LL_2(7)$, then $q=p\equiv 3,5,6\pmod 7$ and $q\neq 5.$ We note that $|H|\le |K|\le 168$, if $H$ is a maximal subfield subgroup then $H\cong \PSU_3(2)$ since $|\PSU_3(3)|\ge K$. 
   Then $K$ is a $2,3$ group of $\LL_2(7)$ since $\mathrm{Spec}(K)\subseteq \mathrm{Spec}(H)$, hence we conclude that $|K|<24<|H|$, a contradiction.
   Similarly, we conclude that $H$ is not a maximal subgroup in Aschbacher class $\mathcal{C}_6$. 
   If $H$ is a maximal subgroup in Aschbacher class $\mathcal{I}$, then since $|K|\ge |H|$, we conclude that both $H$ and $K$ are isomorphic to $\LL_2(7),$ since $H$ and $K$ are not conjugate in $G$, we conclude $3$ divides $q+1$. \textbf{The remaining read   about how $c$ comes.}
   For the remaining cases, by (\ref{order}) we conclude that $$|\LL_2(7)|\ge|K|\ge |H|\ge |\frac{q^2-q+1}{(3,q+1)}{:}3|,$$
   and hence $q=3$ we leave it to Magma.
   
   If $K$ is contained in a maximal subgroup isomorphic to $A_6$, then $q=p\equiv 11,14 \pmod {15}$ and $H$ is isomorphic to $\LL_2(7)$ or $A_6$ since $|K|\ge |H|$. If $H$ is a maximal $\mathcal{I}$ subgroup isomorphic to $\LL_2(7)$, then since $\mathrm{Spec}(K)\subseteq \mathrm{Spec}(H)=\{1,2,3,4,7\}$, then $K$ is a $2,3$ subgroup of $A_6$ and hence $|K|\le 36 <|H|$, a contradiction. 
   If $H$ is a maximal subgroup isomorphic to $A_6$, then \textbf{read about how $c$ comes}.
   Similarly as the argument when $K$ is contained in a maximal subgroup isomorphic to $\LL_2(7)$, we conclude that $H$ is not a subfield subgroup of a maximal subgroup in Aschbacher class $\mathcal{C}_6$, and by (\ref{order}) we have that 
   $$|A_6|\ge |K|>|H|\ge |\frac{q^2-q+1}{(3,q+1)}{:}3|,$$
   and we conclude that $q=11$, $K=A_6$ and $H\cong 37{:}3$ or $H\cong 12^2/3{:}S_3$, but then $5\in \mathrm{Spec}(K)\backslash \mathrm{Spec}(H)$, a contradiction.
   If $K$ is contained in a maximal subgroup isomorphic to  $A_6.2$ or $A_7$, then $q=5$ and we leave it to Magma.
   
   If $H$ is a maximal $\mathcal{I}$ subgroup isomorphic to $\LL_2(7)$, then $q=p\equiv 3,5,6\pmod 7$, and $\mathrm{Spec}(K)\subseteq \mathrm{Spec}(H)=\{1,2,3,4,7\}.$
   Assume $K$ is solvable, and let $M$ be the socle of $K$, then $M$ is elementary abelian $r$-group, where $r\in\{2,3,7\}$ since $\mathrm{Spec}(K)\subseteq \mathrm{Spec}(H).$
   If $r=2$, by Lemma (\ref{Syl-2}) we conclude that $M\cong 2^2$ or $M\cong 2$. Since $K$ does not contain an element of order $6$ or $17$, we conclude that $C_{K}(M)$ is a $2$-group of exponent $4$ and hence $|C_{K}(M)|\le 16$ by Lemma (\ref{Syl-2}).
   As a result, we conclude that $|K|\le |\Aut(M)||C_{K}(M)|\le 96< |H|$, a contradiction.
   If $r=3$, we conclude that $M\cong 3$ or $M\cong 3^2$ by Lemma (\ref{Syl-3}), since $K$ does not contain an element of order $6$ or $21$, we conclude that $C_{K}(M)$ is a $3$-group of exponent $3$, and $|C_{K}(M)|\le 9$ by Lemma (\ref{Syl-3}). 
   On the other hand, we have $K/C_{K}(M)\lesssim \GL_2(3)$ and $\mathrm{Spec}(K/C_{K}(M))\subseteq\{1,2,3,4,7\},$   we conclude that $K/C_K(M)$ is isomorphic to a subgroup of one of $Q_8,D_8$ or $S_3$ by Magma and hence $|K|\le |C_{K}(M)||\Aut(M)|<|H|.$
   If $r=7$, we conclude that $M\cong 7$ or $M\cong 7^2$ by Lemma (\ref{Syl-r}), similarly $C_{K}(M)$ is a $7$-group since $K$ does not contain elements of order $14$ or $21$ and hence $C_{K}(M)\lesssim 7^2$ by Lemma (\ref{Syl-r}).
   If $M\cong 7$, then $|K/C_{K}(M)|\le 6$ and hence $|K|\le |C_{K}(M)||\Aut(M)|<|H|$.
   If $M\cong 7^2$, then $M=C_{K}(M)$ is the Sylow-$7$ subgroup of $G$ and $K\le N_G(M)\cong (q+1)^2/(3,q+1){:}S_3$, we write $N_{G}(M)$ as $N{:}S$ where $N\cong (q+1)^2/(2,q+1)$ and $S\cong S_3.$
   We conclude $K/C_{K}(M)\lesssim N_{G}(M)/C_{G}(M)\lesssim S_3$ since $N\le C_{G}(M)$, on the other hand, each involution not in the center of $\GL_2(7)$ fixes some vector in $\FF_{7}^2$ and hence if $K/C_{K}(M)=S_3$, then there exist $k\in K$ and $m\in M$ such that $k^2=1$ and $km=mk$, and $km$ is of order $14$, a contradiction.
   Therefore, we conclude that $|K|\le |3C_{K}(M)|<|H|.$
   If $K$ is non-solvable, then $K$ contains a simple factor, since a non-abelian simple group with $3$ prime factors
   \textbf{Continue...}
   
   
   If $H$ is a maximal $\mathcal{I}$ group isomorphic to $A_6$, then $\mathrm{Spec}(K)\subseteq \mathrm{Spec}(H)=\{1,2,3,4,5\}$. 
   Assume $K$ is solvable and let $M$ be the socle of $K$, since $\mathrm{Spec}(K)\subseteq\{1,2,3,4,5\},$ then $M$ is an elementary abelian $r$-group for $r\in \{2,3,5\}$, by the same argument when $H\cong \LL_2(7)$, we conclude that $|K|<|H|$.
   
   If $H$ is a maximal $\mathcal{I}$ subgroup isomorphic to $A_6.2$ or $A_7$, we conclude that $q=5$ and we leave all the things to Magma.
  
 \end{document}